\documentclass[specialist,subf,href,colorlinks=true
%,times        % шрифт Times как основной
%,fixint=false % отключить прямые знаки интегралов
]{disser}
\usepackage[
  a4paper, mag=1000, includefoot,
  left=3cm, right=1.5cm, top=2cm, bottom=2cm, headsep=1cm, footskip=1cm
]{geometry}
\usepackage[T2A]{fontenc}
\usepackage[utf8x]{inputenc}
\usepackage[english,russian]{babel}
\usepackage{tikz}
\usepackage{listings}
\graphicspath{{images/}}
\definecolor{light-gray}{gray}{0.95}
\lstset{
  basicstyle=\footnotesize,
  backgroundcolor=\color{light-gray}
}
\begin{document}
\tableofcontents % это оглавление, которое генерируется автоматически

\intro

Достижения в области биологических наук наряду с нарастанием объемов доступной для использования информации повышают необходимость в интеграции разрозненных источников данных. К ним относятся и интернет, и ведомственные, и корпоративные системы:
\begin{enumerate}
\item Медстатистика
\item Результаты клинических испытаний
\item Электронная история болезни
\item Фармацевтика
\item Разработка лекарственных средств
\end{enumerate}

В настоящее время в России разработано и используется большое количество разрозненных медицинских информационных систем, разнообразных баз данных с описанием лекарственных препаратов, результатами научных исследований; написано множество научных трудов и статей в области здравоохранения (медицина, фармацевтика, медицинское страхование и др.), хранящихся в специализированных электронных библиотеках в различных форматах. Однако эффективные механизмы извлечения из таких источников знаний, хранения и предоставления к ним широкого доступа отсутствуют.

Требуется фундаментальный сдвиг от единичных попыток интеграции к единой функциональной области. Для решения этой проблемы разработан стандарт публикации данных Linked Data. Одним из важнейших достоинств этой технологии является ее открытость - возможность объединения в общую семантическую сеть распределенных семантических хранилищ, созданных различными организациями (органы управления здравоохранением, ВУЗы, НИИ, МО) и профессиональными сообществами (ассоциации кардиологов, анестезиологов, медицинских IT-специалистов и др.) на основе единых открытых стандартов. Как показывает международный опыт, это позволяет системе саморазвиваться, постоянно пополняя количество доступных знаний и повышая их качество.

Межресурсные ссылки дают исследователям возможность перемещаться между источниками данных и открывать связи, которые не были замечены ранее. Существуют универсальные инструменты, такие как семантические веб-браузеры и поисковые движки, которые могут использоваться для задач представления и поиска данных.

Основная цель моей работы - это создание семантического хранилища медицинских знаний.

Для достижения поставленной цели решались следующие задачи:
\begin{enumerate}
\item Скачивание и парсинг информации с ресурса Webapteka
\item Разработка онтологии лекарственных препаратов
\item Конвертация html данных в rdf представление
\item Разработка SPARQL-запросов для извлечения информации и выявления дополнительных связей в RDF-хранилище.
\item Разработка пользовательского интерфейса
\item Кеширование элементов приложения для повышения производительности
\end{enumerate}

\chapter{Семантическая сеть}
\section{Введение в семантические сети}

Семантическая сеть (англ. Semantic Web) — это набор технологий, позволяющих представлять информацию в виде пригодном для машинной обработки: RDF, OWL, SPARQL. RDF используется для представления информации, SPARQL - для доступа к ней, OWL - добавляет метаинформацию, связи между концептами.

В RDF вся информация представляется в виде триплетов: субъект, предикат, объект. Триплеты по форме похожи на простое предложение.  Например:
\par Субъект: Александр
\par Предикат: Имеет пол
\par Объект: Мужской
\\Триплет может быть выражен в виде графа
\\
\par \begin{tikzpicture}[->, shorten >=1pt,auto,node distance=5cm,
  thick,main node/.style={circle,fill=blue!20,draw}]

  \node[main node] (2) {Александр};
  \node[main node] (4) [right of=2] {Мужской};

  \path[every node/.style={font=\sffamily\small}]
    (2) edge node {имеет пол} (4);
\end{tikzpicture}

Субъекты и объекты могут быть представлены URI, либо литералом. URI - это уникальный идентификатор, который обозначает сущность: например URI для собаки может быть таким \textit{'http://example.ru/animals/dog'}. Литерал - это просто строка, например 'Jack Nickolson', с возможными добавлениями, указывающими язык, тип данных (поддерживаемый XML, такие как integer и datetime). В идеале каждая между сущностями и URI составлено взаимно однозначное соответствие: каждый URI принадлежит только одной сущности и каждая сущность имеет только один URI.
Для обозначение предикатов всегда используются URI.

Использование URI в RDF облегчает нахождение документов, связанных с сущностью. Например, если кто-то(или чья-то программа) ищет информацию о собаках, то ему надо искать все триплеты содержащие URI \textit{http://www.example.ru/animals/dog}.

RDF-документ представляет собой набор триплетов. Его можно выразить в виде графа, если представить URI как вершины, а предикаты как ребра графа.

\par \begin{tikzpicture}[->, shorten >=1pt,auto,node distance=5cm,
  thick,main node/.style={circle,fill=blue!20,draw}]
  \node[main node] (2) {Александр};
  \node[main node] (3) [below of=2] {Казань};
  \node[main node] (1) [right of=3] {Татарстан};
  \node[main node] (4) [right of=2] {Мужской};

  \path[every node/.style={font=\sffamily\small}]
    (2) edge node {имеет пол} (4)
	edge node {живет в} (3)
    (3) edge node {столица} (1);
\end{tikzpicture}

Таким образом можно построить граф неограниченного размера.
RDF как язык для хранилища имеет ряд преимуществ:
\begin{enumerate}
\item Простое агрегирование данных. Необходимо только добавить триплеты с указанием связи между сущностями.
\item Использование URI дает возможность объединять информацию о сущности с нескольких источников данных.
\item Поскольку RDF не имеет жестких, заведомо заданных требований к структуре данных, к наличию или отсутствию свойства, повышается плотность хранения информации.
\item RDF предлагает единый язык для представления практически любого знания.
\end{enumerate}
\section{OWL}
Технологии Semantic Web дают возможность выводить новые факты из базовых фактов, хранящихся в RDF. OWL добавляет к RDF информацию о классах, типах, логических зависимостях, доменов у свойств, пространстве возможных значений. 
\section{SPARQL}
SPARQL - язык запросов к RDF-хранилищу. SPARQL, как и большинство языков такого типа, содержит переменные в тексте запроса, в которые подставляются извлеченные данные.
Запрос вида
\begin{lstlisting}
SELECT ?x WHERE {
?x <http://www.example.com/has-gender> <http://www.example.com/male> . 
}
\end{lstlisting}
найдет все триплеты с указанным предикатом и объектом(ИМЕЕТ ПОЛ, МУЖСКОЙ) и вернет список субъектов. Информация возвращается в XML-формате.

Запрос может быть построен из нескольких триплетов. В следующем примере кода две конструкции, которые должны вернуть всех людей мужского пола.
\begin{lstlisting}
SELECT ?x WHERE {
?x <http://www.example.com/has-gender> <http://www.example.com/male> .
?x <http://www.example.com/has-species> <http://www.example.com/human> .
}
\end{lstlisting}
Это тип запросов основной в SPARQL. Хотя SPARQL беднее по функциональности чем SQL, он поддерживает схожий функционал для уточнения запроса: сортировка результатов, получение подмножества результатов, удаление дубликатов и т.д.
Следующий запрос вернет всех мужчин, которые имеют больше, чем 20 книг и, если имеется информация о предпочтениях в еде, она вернется тоже.
\begin{lstlisting}
PREFIX ex: <http://www.example.com/>
SELECT ?x ?foods WHERE {
?x ex:has-gender ex:male .
?x ex:has-species ex:human .
?x ex:has-book-count ?bookcount .
FILTER (?bookcount < 20)
}
OPTIONAL {
?x ex:likes-food ?foods .
}
}
\end{lstlisting}
Преимущества, которые могут быть получены за счет использования этого языка запросов понятны: человек или компьютер могут соединиться с любым открытым репозиторием, сделать очень специфичный запрос и получить машино-обрабатываемые данные.
\chapter{Сравнение RDF с другими моделями хранилищ}
В любом хранилище данных, доступ к информации осуществляется в соответствии с некоторой моделью, логической концепцией. В этой главе описываются модели хранения данных, использующиеся в настоящее время. Исследуются сходства RDF-модели с остальными и определяется, в какой степени подходы для традиционных баз данных применимы к RDF-хранилищам.
\section{Key-Value хранилища}
\section{Реляционные хранилища}
Эта модель была предложена в 1970г. Э.Коддом. В этом подходе теория множеств и логика предикатов используются для определения логической структуры хранилища данных и операций, которые могут быть к нему применены. В часности, разделяются логическая структура и физическая. СУБД может выбрать любой способ физического хранения данных, но то, как информация отображается пользователю, остается неизменным.

Реляционная модель описывает данные в терминах реляций, состоящих из неограниченного числа кортежей и аттрибутов. Реляции в целом схожи с таблицами, состоящих из строк и столбцов. Каждый кортеж является уникальным (ведь не имеет смысла один и тот же факт дважды).  Запросы к СУБД пишутся на декларативном языке, позволяющим пользователям указать, какие данные они хотят получить, не заставляя их указывать, каким способом это сделать. Как правило, это ответственность СУБД сделать запрос как можно более быстрым. Компонент, который выполняет оптимизацию, называется оптимизатором запросов.

Реляционная модель предназначена для поддержки запросов с перекрестными ссылками между блоками данных: 
\textit{
  'Получить всех механиков, которые работали над машиной, содержащей деталь X.'
}

\section{RDF хранилище}
Требование к строгому определению структуры базы данных характерны для СУБД, ситуация с RDF принципиально другая, ведь RDF был разработан максимально безструктурным. Как отмечалось ранее, это имеет свои преимущества в условиях доступа к произвольным источникам данных в Semantic Web, а также ситуациях, когда данные имеют неизвестные, постоянно меняющиеся структуры. Однако, отсутствие определенной структуры приводит к трудностям при хранении большого количества триплетов и быстрого выполнения запросов. Современные RDF-хранилища могут хранить на порядок меньше данных чем СУБД.

RDF-хранилище может быть спроецировано в СУБД, оптимизированной для RDF-модели с помощью индексов и других тактик. SPARQL запросы при этом транслируются в SQL. В самой простой реализации, RDF представляется с помощью одной таблицы. Если пользоваться правилом нормализации, то появляются еще таблицы, хранящие URI и литералы, а триплеты используют внешние ключи, указывающие на записи в других таблицах. 

К несчастью, гибкость RDF представляет собой барьер на пути создания сложных, выразительных представлений. Легкость изменения информации в RDF приводит к сверх-трудностям при создании эффективных схем хранения. Это можно назвать основным отличием RDF от других представлений.

\begin{tabular}{p{0.22\linewidth}|p{0.25\linewidth}|p{0.2\linewidth}|p{0.2\linewidth}}
\hline
& Рекомендованное\ использование & Структура хранимой информации & Запросы  \\
\hline
RDF & Произвольное представление знаний & Триплеты & Неизвестный уровень предсказуемости \\
\hline
Реляционные & Поддержка приложений, база данных & Таблицы, предопределенне структуры & Более предсказуемые запросы, оптимизация запросов \\
\hline
Ключ/Значение & Поддержка приложений& Пары ключ-значение & Неизвестный уровень предсказуемости \\
\hline
\end{tabular}

Таким образом при построении RDF-хранилища необходимо предусмотреть будущие проблемы, связанные с большим количеством триплетов, изменчивостью структуры данных. При этом интерфейс приложения должен быть понятен пользователю. 

\chapter{Разработка RDF-хранилища}

\section{Построение онтологии}
При построении RDF-хранилища, в первую очередь разрабатывается онтология - описание схемы данных хранилища на языке OWL. Уже существует множество полезных онтологий, части которых можно использовать в своей онтологии либо создавать на них ссылки.(проект http://swoogle.umbc.edu/ позволяет искать онтологии). Мощным средством для построения онтологий является Protege.

Основной класс в онтологии проекта - это Drug(лекарство). Большинство остальных классов образуют объектная часть предикатов.

Список свойств класса Drug:
\begin{enumerate}
 \item имеет международное название
 \item имеет торговое наименование
 \item принадлежит нескольким категориям
 \item имеет множество форм применения
 \item имеет фармакологию
 \item имеет множество показаний
 \item имеет множество противопоказаний
 \item имеет множество побочных эффектов
 \item имеет дозировку
 \item взаимодействует с другими лекарствами и группами лекарств
 \item содержит лекарственные ингридиенты
 \item имеет специальные указания
\end{enumerate}

\section{Выбор архитектуры и инструментов}

Ruby - мощный и гибкий язык. Он предоставляет лаконичный синтаксис, возможность изменять классы и методы во время выполнения программы, на нем написано множество библиотек, существенно облегчающих построение Web-приложения. Руководствуясь этими соображениями, RDF-хранилище будет писаться на Ruby.

Существует несколько библиотек для работы с RDF-хранилищами, реализованных на Ruby. Основные отличия между ними – поддержка функций вывода, степень поддержки вывода для OWL, возможности использования в качестве точки доступа SPARQL, веб-доступ, возможность хранения четверок вместо триплетов, а так же поддержка хранилища языками программирования (наличие модулей).

Кроме того, важным критерием является способность проецировать схему субъект-предикат-объект на экземпляр-свойство-значение, таким образом используя объектно-ориентированное программирование. Существуют, однако, глубинные различия объектов из ООП языков и RDF-объектами. Особенность RDF-сущностей в том, что они имеют URI, и они могут экземплярами нескольких классов. Для преодоления этой проблемы используется общая модель данных для ООП и RDF. В этой модели сущности могут иметь методы и свойства, должны иметь URI и могут быть экземплярами нескольких классов. Таким образом сочетаются возможности ООП и RDF.

Для использования в проекте, выбрана библиотека RDFrb. Она предоставляет наиболее обширный диапазон возможностей, и, что важно с иследовательской точки зрения, предоставляет абстрактный класс для переопределения функций хранилища.

Web-сервер будет базироваться на Ruby on Rails - фреймворке, написанном на Ruby, предоставляющем возможности xml-сериализации, html-парсинга, кеширования, REST-архитектуры.
\section{RDFrb}
\section{Spira}
\section{Sparql-Ruby}

\chapter{Заполнение RDF хранилища}
\section{Парсинг интернет-ресурса Web-аптека}
\chapter{Создание SPARQL-точки}
\chapter{Построение web-интерфейса}



\end{document}
