\documentclass[specialist,subf,href,colorlinks=true
%,times        % шрифт Times как основной
%,fixint=false % отключить прямые знаки интегралов
]{disser}
\usepackage[
  a4paper, mag=1000, includefoot,
  left=3cm, right=1.5cm, top=2cm, bottom=2cm, headsep=1cm, footskip=1cm
]{geometry}
\usepackage[T2A]{fontenc}
\usepackage[utf8x]{inputenc}
\usepackage[english,russian]{babel}
\usepackage{tikz}
\graphicspath{{images/}}
\begin{document}
\tableofcontents % это оглавление, которое генерируется автоматически

\intro

Достижения в области биологических наук наряду с нарастанием объемов доступной для использования информации повышают необходимость в интеграции разрозненных источников данных. К ним относятся и интернет, и ведомственные, и корпоративные системы:
\begin{enumerate}
\item Медстатистика
\item Результаты клинических испытаний
\item Электронная история болезни
\item Фармацевтика
\item Разработка лекарственных средств
\end{enumerate}

В настоящее время в России разработано и используется большое количество разрозненных медицинских информационных систем, разнообразных баз данных с описанием лекарственных препаратов, результатами научных исследований; написано множество научных трудов и статей в области здравоохранения (медицина, фармацевтика, медицинское страхование и др.), хранящихся в специализированных электронных библиотеках в различных форматах. Однако эффективные механизмы извлечения из таких источников знаний, хранения и предоставления к ним широкого доступа отсутствуют.

Требуется фундаментальный сдвиг от единичных попыток интеграции к единой функциональной области. Для решения этой проблемы разработан стандарт публикации данных Linked Data. Одним из важнейших достоинств этой технологии является ее открытость - возможность объединения в общую семантическую сеть распределенных семантических хранилищ, созданных различными организациями (органы управления здравоохранением, ВУЗы, НИИ, МО) и профессиональными сообществами (ассоциации кардиологов, анестезиологов, медицинских IT-специалистов и др.) на основе единых открытых стандартов. Как показывает международный опыт, это позволяет системе саморазвиваться, постоянно пополняя количество доступных знаний и повышая их качество.

Межресурсные ссылки дают исследователям возможность перемещаться между источниками данных и открывать связи, которые не были замечены ранее. Существуют универсальные инструменты, такие как семантические веб-браузеры и поисковые движки, которые могут использоваться для задач представления и поиска данных.

Основная цель моей работы - это создание семантического хранилища медицинских знаний.

Для достижения поставленной цели решались следующие задачи:
\begin{enumerate}
\item Скачивание и парсинг информации с ресурса Webapteka
\item Разработка онтологии лекарственных препаратов
\item Конвертация html данных в rdf представление
\item Разработка SPARQL-запросов для извлечения информации и выявления дополнительных связей в RDF-хранилище.
\item Разработка пользовательского интерфейса
\item Кеширование элементов приложения для повышения производительности
\end{enumerate}

\chapter{Семантическая сеть}
\section{Введение в семантические сети}

Семантическая сеть (англ. Semantic Web) — это набор технологий, позволяющих представлять информацию в виде пригодном для машинной обработки: RDF, OWL, SPARQL. RDF используется для представления информации, SPARQL - для доступа к ней, OWL - добавляет метаинформацию, связи между концептами.

В RDF вся информация представляется в виде триплетов: субъект, предикат, объект. Триплеты по форме похожи на простое предложение.  Например:
\par Субъект: Александр
\par Предикат: Имеет пол
\par Объект: Мужской
\\Триплет может быть выражен в виде графа
\\
\par \begin{tikzpicture}[->, shorten >=1pt,auto,node distance=5cm,
  thick,main node/.style={circle,fill=blue!20,draw}]

  \node[main node] (2) {Александр};
  \node[main node] (4) [right of=2] {Мужской};

  \path[every node/.style={font=\sffamily\small}]
    (2) edge node {имеет пол} (4);
\end{tikzpicture}

Субъекты и объекты могут быть представлены URI, либо литералом. URI - это уникальный идентификатор, который обозначает сущность: например URI для собаки может быть таким \textit{'http://example.ru/animals/dog'}. Литерал - это просто строка, например 'Jack Nickolson', с возможными добавлениями, указывающими язык, тип данных (поддерживаемый XML, такие как integer и datetime). В идеале каждая между сущностями и URI составлено взаимно однозначное соответствие: каждый URI принадлежит только одной сущности и каждая сущность имеет только один URI.

\end{document}
